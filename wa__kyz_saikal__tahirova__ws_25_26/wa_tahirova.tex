\documentclass[a4paper,12pt]{article}

% --- Sprach- und Kodierungseinstellungen ---
\usepackage[utf8]{inputenc}
\usepackage[T1]{fontenc}
\usepackage[ngerman]{babel}

% --- Typografie ---
\usepackage{lmodern}
\usepackage{microtype}
\usepackage{csquotes} % Korrekte deutsche Anführungszeichen
\usepackage{setspace}
\onehalfspacing

% --- Seitenlayout ---
\usepackage[left=2.5cm,right=2.5cm,top=3cm,bottom=3cm]{geometry}
\usepackage{indentfirst}
\setlength{\parindent}{1.25em} % Отступ первой строки
\setlength{\parskip}{0pt}      % Без дополнительного вертикального расстояния

% --- Grafiken ---
\usepackage{graphicx}
\graphicspath{{./img/}} 

% --- Farben, Links, URLs ---
\usepackage{xcolor}
\usepackage{hyperref}
\usepackage{url}
\urlstyle{same}
\hypersetup{
    pdftitle={Nutzung von Sprachmodellen zur Verbesserung von Bezeichnern in Java-Programmen},
    pdfauthor={Kyz Saikal Tahirova},
    pdfsubject={Wissenschaftliche Arbeit},
    pdfkeywords={Sprachmodelle, Java, Bezeichner, Programmierung},
    colorlinks=true,
    linkcolor=black,
    citecolor=black,
    urlcolor=blue,
    pdfpagemode=UseOutlines
}

\usepackage{enumitem}
\setlist{itemsep=3pt, topsep=3pt, parsep=0pt, partopsep=0pt}

% --- Mathe und Tabellen ---
\usepackage{amsmath,amssymb,booktabs}

% --- Quellcode ---
\usepackage{listings}
\lstdefinestyle{javaStyle}{
    language=Java,
    basicstyle=\ttfamily\small,
    keywordstyle=\color{blue!70!black}\bfseries,
    stringstyle=\color{red!60!black},
    commentstyle=\color{gray},
    numbers=left,
    numberstyle=\tiny\color{gray},
    stepnumber=1,
    showstringspaces=false,
    tabsize=4,
    breaklines=true,
    frame=single,
    inputencoding=utf8,
    extendedchars=true,
    literate={ä}{{\"a}}1 {ö}{{\"o}}1 {ü}{{\"u}}1 {ß}{{\ss}}1
}
\lstset{style=javaStyle}

% --- Verzeichnisse ---
\usepackage[nottoc,notlof,notlot]{tocbibind}

% ===============================
%           Dokument
% ===============================
\begin{document}

% --- Titelseite ---
\begin{titlepage}
    \centering
    \includegraphics[width=0.7\textwidth]{WHZ-Logo.jpg}\\[1cm]

    {\Large\textbf{Westsächsische Hochschule Zwickau}}\\[1ex]
    {\large Fakultät für Physikalische Technik und Informatik}\\[4cm]

    {\LARGE\textbf{Wissenschaftliche Arbeit}}\\[1cm]

    {\Large\textbf{Thema:}}\\[0.5cm]
    {\Large Nutzung von Sprachmodellen zur Verbesserung von Bezeichnern in Java-Programmen}\\[3cm]

    \begin{tabular}{rl}
        Vorgelegt von: & Kyz Saikal Tahirova \\
        Matrikel-Nr.:  & 66110               \\
        Studiengang:   & Informatik          \\
        Betreuer:      & Prof.\ Dr.\ Laue    \\
        Abgabetermin:  & xx.xx.2025          \\
    \end{tabular}
\end{titlepage}

\setcounter{page}{2}

\clearpage

% --- Inhaltsverzeichnis ---
\tableofcontents
\clearpage

% --- Hauptteil ---
\section*{Kapitel 1}
\section{Einleitung}

Sprachmodelle werden in der Softwareentwicklung immer häufiger eingesetzt, um
Entwickler bei Analyse- und Review-Aufgaben zu unterstützen. Ein wichtiger
Anwendungsbereich ist die Bewertung von Bezeichnern im Quellcode. Klare und
verständliche Namen vereinfachen die Wartung, während verschwommene Bezeichner
zu Fehlern und erhöhtem Aufwand führen
\cite{identifiernamesinfluencecodequality}.

Diese Arbeit untersucht die Zuverlässigkeit moderner Sprachmodelle bei der
Einschätzung von Java-Bezeichnern. Bewertet werden Verständlichkeit,
semantische Angemessenheit, Einhaltung etablierter Namenskonventionen
\cite{oracle} sowie die Erkennung linguistischer Anti-Muster
\cite{arnaoudova2014}. Unter linguistischen Anti-Mustern versteht man
systematische sprachliche Probleme in Bezeichnern, die Klarheit und Lesbarkeit
beeinträchtigen.

Ziel der Untersuchung ist es zu prüfen, ob Sprachmodelle hilfreiche und
konsistente Hinweise zur Verbesserung von Bezeichnern liefern können. Hierfür
werden mehrere Hinweis-Varianten entwickelt und schrittweise verfeinert. Die
Analyse erfolgt anhand von Java-Beispielen mit sowohl problematischen als auch
korrekten Bezeichnern, um die Erkennung tatsächlicher Verstöße und das
Auftreten unbegründeter Warnungen („False Positives“) zu bewerten. Frühere
Arbeiten zeigen bereits, dass LLM-generierter Code zu semantischen
Fehlbenennungen und strukturellen Qualitätsproblemen neigt
\cite{llmsmells2025,llmmethodnames2025,aitomaticprogramming}.

Die zentrale Forschungsfrage lautet:
\begin{quote}
    Wie zuverlässig können Sprachmodelle Java-Bezeichner hinsichtlich Verständlichkeit, Bedeutung, Konventionen und linguistischer Anti-Muster bewerten?
\end{quote}

Die Arbeit ist in mehrere Kapitel gegliedert:

\begin{enumerate}
    \item \textbf{Kapitel 1 (Einleitung).} Vorstellung des Themas, Zielsetzung, Relevanz von Sprachmodellen in der Softwareentwicklung sowie die Forschungsfrage.
    \item \textbf{Kapitel 2 (Theoretische Hintergrund).} Überblick über Java-Bezeichner, Namenskonventionen und linguistische Anti-Muster.
    \item \textbf{Kapitel 3 (Methodik und Durchführung).} Aufbau der Untersuchung, Entwicklung der Hinweis-Varianten sowie Beschreibung der verwendeten Java-Beispiele. \textit{Eine automatische Pipeline wird nur theoretisch erwähnt.}
    \item \textbf{Kapitel 4 (Evaluation).} Auswertung der Ergebnisse und systematische Bewertung der Antworten der Sprachmodelle.
    \item \textbf{Kapitel 5 (Diskussion und Schluss).} Grenzen der Methode, Verbesserungsideen, Ausblick auf weitere Forschung.

\end{enumerate}

Hinweis: In dieser ersten Version der Arbeit wird der automatische
Pipeline-Ansatz nur theoretisch erläutert, da die Umsetzung zum jetzigen
Zeitpunkt noch nicht erfolgt ist. In einer späteren Version könnte diese
Funktion ergänzt werden, um den gesamten Prozess der Bezeichnerverbesserung
automatisch abzubilden.

\newpage
\section*{Kapitel 2}
\section{Theoretische Hintergrund}
Dieser Abschnitt bietet einen Überblick über Java-Bezeichner, etablierte
Namenskonventionen sowie linguistische Anti-Muster, die als Grundlage für die
spätere Analyse dienen. Die Relevanz dieser Konzepte ergibt sich aus
empirischen Studien, die zeigen, dass Bezeichner einen erheblichen Einfluss auf
die Verständlichkeit und Qualität von Quellcode haben
\cite{identifiernamesinfluencecodequality}.

\subsection{Java-Bezeichner}

Java-Bezeichner sind die Namen von Variablen, Methoden, Klassen,
Schnittstellen, Konstanten und Paketen. Sie sollen die Funktion und den Zweck
des Codes verständlich machen. Die Studie von Butler et al.
\cite{identifiernamesinfluencecodequality} zeigt, dass unklare, inkonsistente
oder mehrdeutige Namen mit geringer Wartbarkeit, erhöhter Komplexität und
häufigeren Warnungen in statischen Analysewerkzeugen verbunden sind. Im
Gegensatz dazu können klare Namen ein einfacher Indikator für eine hohe
Qualität des Quellcodes sein.

\subsection{Namenskonventionen in Java}

Die offiziellen Java-Konventionen \cite{oracle} empfehlen konsistente
Schreibweisen und semantisch präzise Nomenphrasen. Dazu gehören:

\begin{itemize}
    \item \textbf{Klassen und Schnittstellen.} Großschreibung jedes Wortanfangs (CamelCase), z.\,B. \texttt{CustomerAccount}, \texttt{ PaymentProcessor}. Allgemein sollten aussagekräftige Nomenphrasen verwendet werden, etwa \texttt{EnrolledStudents} oder \texttt{ NumberOfValidCreditCards}.
    \item \textbf{Methoden.} Verben oder Verbphrasen wie \texttt{ calculateTotal()} oder \texttt{validateInput()}.
    \item \textbf{Attribute/Variablen.} camelCase und präzise Bedeutung, z.\,B. \texttt{userName}, \texttt{itemCount}.
\end{itemize}

Ein Name soll beschreiben, was ein Element speichert oder ausführt. Unklare
Begriffe wie \texttt{tmp}, \texttt{data} oder \texttt{handle} lassen zu viel
Interpretationsspielraum und führen schnell zu Missverständnissen.

\subsection{Linguistische Anti-Muster}

Linguistische Anti-Muster (Linguistic Anti-Patterns) wurden erstmals
systematisch von Arnaoudova et al. beschrieben \cite{arnaoudova2014}. Sie
bezeichnen wiederkehrende sprachliche Probleme in Bezeichnern, Kommentaren oder
Signaturen, bei denen Name und tatsächliches Verhalten nicht übereinstimmen.
Dies führt zu kognitiven Brüchen und erschwert das Programmverständnis.

Die Anti-Muster lassen sich in zwei Gruppen einteilen.

Methodenbezogene Anti-Muster:

\begin{enumerate}
    \item \textbf{Kategorie A: „tut mehr als der Name sagt“}
          Die Methode führt zusätzliche Aktionen aus.
          Beispiel: \texttt{getUser()} validiert oder speichert Daten.

    \item \textbf{Kategorie B: „sagt mehr als sie tut“}
          Der Name verspricht Funktionalität, die nicht implementiert ist.
          Beispiel: \texttt{validate()} besitzt den Rückgabetyp \texttt{void}.

    \item \textbf{Kategorie C: „tut das Gegenteil“}
          Name und Verhalten widersprechen sich.
          Beispiel: \texttt{disable()} erzeugt einen \texttt{EnableState}.
\end{enumerate}

Attributbezogene Anti-Muster:
\begin{enumerate}
    \setcounter{enumi}{3}
    \item \textbf{Kategorie D: enthält mehr als der Name sagt}
          Beispiel: ein als Prädikat benannter Bezeichner, dessen Typ jedoch kein \texttt{boolean} ist.

    \item \textbf{Kategorie E: Name sagt mehr als enthalten ist}
          Beispiel: \texttt{users} als Name für ein einzelnes Objekt.

    \item \textbf{Kategorie F: Name und Inhalt widersprechen sich}
          Beispiel: Attributname und Typ bilden Antonyme.
\end{enumerate}

\subsection{Große Sprachmodelle (LLMs)}

Sprachmodelle werden zunehmend zur Bewertung von Quellcode verwendet
\cite{aitomaticprogramming}. Studien zeigen jedoch, dass LLMs zwar Muster
erkennen können, aber gleichzeitig zu einer oberflächlichen Analyse neigen, insbesondere wenn nur Identifikatoren geändert oder semantisch irrelevante Änderungen vorgenommen werden \cite{llmmaintainablereliable2025}. Darüber hinaus zeigen mehrere Studien eine hohe Rate an Fehlalarmen („False Positives”) und ungenauen Bewertungen \cite{llmsmells2025, llmdetection2025}.

In dieser Arbeit wird untersucht, wie zuverlässig Sprachmodelle Java-Bezeichner bewerten können und in welchen Fällen weiterhin menschliches Eingreifen erforderlich ist.

\newpage
\section*{Kapitel 3}
\section{Methodik und Durchführung}

\subsection{Auswahl der Sprachmodelle}

Für die Untersuchung wurden drei verschiedene Modelle über HAWKI\footnote{HAWKI
    dient als Schnittstelle zu OpenAI GPT-4o, Meta LLaMA 3.1 70B Instruct, Alibaba
    Qwen 2.5 72B Instruct Modellen} abgefragt:

\begin{itemize}
    \item OpenAI GPT-4o,
    \item Meta LLaMA 3.1 70B Instruct,
    \item Alibaba Qwen 2.5 72B Instruct.
\end{itemize}

\subsection{Erstellung der Test-Beispiele}

\subsubsection{Kategorien nach Arnaoudova}

Basierend auf Arnaoudova et al. \cite{arnaoudova2014} wurden zwei Arten von
Beispielen erstellt:

\begin{itemize}
    \item \textbf{10--15 Beispiele mit linguistischen Anti-Mustern} (alle Kategorien A–F),
    \item \textbf{5--7 korrekt benannte Beispiele} ohne Anti-Muster, um unbegründete Fehlalarme zu erkennen.
\end{itemize}

Jede Kategorie wurde kurz erklärt und durch ein prägnantes Beispiel
illustriert.

\subsubsection{Beispiel-Code}

\textit{Kategorie A1 – „tut mehr als der Name sagt“.} Methode mit \texttt{get} macht mehr als zurückgeben:

\begin{lstlisting}
public class UserManager {
    private User currentUser;

    public User getUser() {
        validateSession();   // Nebenwirkung
        return currentUser;
    }
}
\end{lstlisting}

\textit{Kategorie B3.} Name deutet Rückgabe an, aber \texttt{void}:

\begin{lstlisting}
public class Calculator {
    public void getResult() {
        System.out.println("42");
    }
}
\end{lstlisting}

\textit{Kategorie C1.} Widerspruch zwischen Name und Rückgabewert:

\begin{lstlisting}
public class Config {
    // Problem: disable() gibt ControlEnableState zuruck
    public ControlEnableState disable() {
        return new ControlEnableState(true);
    }
}
\end{lstlisting}

\textit{Kategorie D2.} Prädikatname, Typ nicht Boolean:
\begin{lstlisting}
public class Payment {
    private int isActive; // Typ passt nicht zum Prädikat
}
\end{lstlisting}

\textit{Kategorie E1.} Pluralname bei singular Typ:
\begin{lstlisting}
public class Order {
    private Product products; // Name suggeriert Mehrzahl
}
\end{lstlisting}

\textit{Kategorie F1.} Name und Typ im Gegensatz:
\begin{lstlisting}
public class Mode {
    private boolean disabledState = true;
}
\end{lstlisting}

\subsubsection{Korrekte Beispiele für False-Positive-Tests}

\begin{lstlisting}
public class StudentRecord {
    private int enrolledStudents;
}
\end{lstlisting}

Diese Beispiele enthalten keine Anti-Muster. Ein korrektes Modell darf hier \emph{keine} Warnung ausgeben \cite{llmdetection2025}.

\subsection{Hinweis-Strategien und Iteration}
Die Hinweise wurden iterativ nach Versionen überarbeitet: Version 1 (V1) → Version 2 (V2) → Version 3 (V3) → Version 4 (V4). Das Ziel jeder Iteration war die Qualitätssteigerung der Hinweise (automatischer Überblick) und die Reduzierung der Anzahl von Fehlalarmen. V4 enthält auch eine kurze Definition linguistischer Anti-Muster, damit Modelle diese Kategorien zielorientiert erkennen können.

\paragraph{V1 (Zero-Shot)}
\begin{verbatim}
    Analysieren Sie die Bezeichner im folgenden Java-Code
    und identifizieren Sie mögliche Benennungs- oder Struktur-Anti-Muster.
\end{verbatim}

\paragraph{V2 (Few-Shot)}
\begin{verbatim}
    Beispiele für gute Bezeichner:
    - Klassen: CustomerAccount, PaymentProcessor
    - Methoden: calculateTotal(), validateInput()
    - Attribute: userName, itemCount
    Analysieren Sie nun folgenden Code:
    [CODE]
\end{verbatim}

\paragraph{V3 (Kontextreicher Hinweis)}
\begin{verbatim}
    Analysieren Sie den Code unter Berücksichtigung von:
    1) Java-Konventionen (Oracle)
    2) Konsistenz zwischen Name, Typ und Verhalten
    3) Verständlichkeit und Bedeutung
    Geben Sie Hinweise für Entwickler.
\end{verbatim}

\paragraph{V4 (Erklärung der linguistischen Anti-Muster)}
\mbox{}\\
Da unklar ist, ob Modelle linguistische Anti-Muster kennen \cite{arnaoudova2014}, wird in dieser Version eine kurze Erklärung ergänzt.

\begin{verbatim}
    Linguistische Anti-Muster:
    A: Methode tut mehr als Name sagt
    B: Name verspricht mehr als implementiert
    C: Name und Verhalten sind gegensätzlich
    D-F: Inkonsistenzen bei Attributnamen
    Analysieren Sie nun den Code:
    [CODE]
\end{verbatim}

\subsection{Iteratives Vorgehen}

Der Hinweis wurde iterativ verbessert:

\begin{enumerate}
    \item erste Tests mit Version 1,
    \item Analyse der Fehlbewertungen,
    \item Erweiterung um Beispiele (Version 2),
    \item Hinzufügen von Kontextregeln (Version 3),
    \item Ergänzung der Definitionen von Anti-Mustern (Version 4).
\end{enumerate}

Dieses Vorgehen folgt aktuellen Forschungsergebnissen, die die Bedeutung von Hinweis-Gestaltung und Iteration bei der Codeanalyse hervorheben \cite{llmmethodnames2025, aitomaticprogramming}.

\subsection{Durchführung der Tests}

Für jedes der 10–15 fehlerhaften Beispiele und für jedes der 5–7 korrekten Beispiele wurde jede Hinweis-Version an alle drei Modelle über HAWKI gesendet. Jede Modellantwort wurde protokolliert und anonymisiert abgespeichert.

\subsection{Bewertungskriterien}

Jedes Kriterium wird mit 1-5 Punkten bewertet.

\begin{enumerate}
    \item \textbf{Anti-Muster-Erkennung.} Wurde das Problem erkannt?

    \item \textbf{Korrekturqualität.} Ist der vorgeschlagene neue Name besser? Löst er das Anti-Muster?

    \item \textbf{Konventions-Konformität.} Folgt der neue Code Java-Standards?

    \item \textbf{Semantische Klarheit.} Sind die neuen Namen verständlich?

    \item \textbf{Konsistenz.} Passt alles zusammen?
\end{enumerate}

Zusätzlich wurden Fehlalarme („False Positives“) gesondert untersucht, da aktuelle Forschung zeigt, dass sie bei LLM-basierten Analysen häufig auftreten \cite{llmdetection2025, llmmaintainablereliable2025}.




Das Bewusstsein für linguistische Anti-Muster hilft, Missverständnisse zu
vermeiden und die Lesbarkeit, Verständlichkeit und Wartbarkeit von Software nachhaltig zu verbessern.

% --- Literaturverzeichnis ---
\clearpage
\begin{thebibliography}{99}

    \bibitem{arnaoudova2014}
    Arnaoudova, V.; Di Penta, M.; Antoniol, G. (2014):
    \textit{Linguistic Anti-Patterns: What They Are and How Developers Perceive Them.}
    Empirical Software Engineering.
    Verfügbar unter: \url{https://www.veneraarnaoudova.ca/wp-content/uploads/2014/10/2014-EMSE-Arnaodova-et-al-Perception-LAs.pdf},
    abgerufen am 14.10.2025.

    \bibitem{oracle}
    Oracle Corporation:
    \textit{Naming Conventions.}
    Verfügbar unter: \url{https://www.oracle.com/java/technologies/javase/codeconventions-namingconventions.html},
    abgerufen am 14.10.2025.

    \bibitem{aitomaticprogramming}
    Lyu, M. R.; Rajan, B.; Roychoudhury, A.; Tan, S. H.; Thummalapenta, P.:
    \textit{Automatic Programming: Large Language Models and Beyond.}
    Verfügbar unter: \url{https://dl.acm.org/doi/pdf/10.1145/3708519},
    abgerufen am 15.11.2025.

    \bibitem{identifiernamesinfluencecodequality}
    Butler, S.; Wermelinger, M.; Yu, Y.; Sharp, H. (2010):
    \textit{Exploring the Influence of Identifier Names on Code Quality: An Empirical Study.}
    In: Proceedings of the 14th European Conference on Software Maintenance and Reengineering (CSMR), 15--18 March 2010, Madrid, Spain.
    Verfügbar unter: \url{https://www.researchgate.net/publication/42799923_Exploring_the_Influence_of_Identifier_Names_on_Code_Quality_An_Empirical_Study},
    abgerufen am 26.11.2025.

    \bibitem{llmsmells2025}
    Ghosh Paul, D.; Zhu, H.; Bayley, I. (2025):
    \textit{Investigating the Smells of LLM Generated Code.}
    School of Engineering, Computing and Mathematics, Oxford Brookes University, October 2025.
    Verfügbar unter: \url{https://www.researchgate.net/publication/396223444_Investigating_The_Smells_of_LLM_Generated_Code},
    abgerufen am 26.11.2025.

    \bibitem{llmmaintainablereliable2025}
    Santa Molison, A.; Moraes, M.; Melo, G.; Santos, F.; Assunção, W. K. G. (2025):
    \textit{Is LLM-Generated Code More Maintainable \& Reliable than Human-Written Code?}
    Toronto Metropolitan University; Colorado State University; North Carolina State University, July 2025.
    Verfügbar unter: \url{https://www.researchgate.net/publication/393853113_Is_LLM-Generated_Code_More_Maintainable_Reliable_than_Human-Written_Code},
    abgerufen am 26.11.2025.

    \bibitem{llmmethodnames2025}
    Akram, W.; Jiang, Y.; Zhang, Y.; Khan, H. A.; Liu, H. (2025):
    \textit{LLM-Based Method Name Suggestion with Automatically Generated Context-Rich Prompts.}
    Beijing Institute of Technology; Peking University.
    Verfügbar unter: \url{https://www.researchgate.net/publication/392855381_LLM-Based_Method_Name_Suggestion_with_Automatically_Generated_Context-Rich_Prompts},
    abgerufen am 27.11.2025.

    \bibitem{llmdetection2025}
    Andrade, R.; Torres, J.; Ortiz-Garcés, I. (2025):
    \textit{Enhancing Security in Software Design Patterns and Antipatterns: A Framework for LLM-Based Detection.}
    In: Electronics, 14(3) (2025), Artikel Nr. 586.
    DOI: \url{https://doi.org/10.3390/electronics14030586}.
    Submission received: 16.11.2024; revised: 21.01.2025; accepted: 28.01.2025; published: 01.02.2025.

\end{thebibliography}

\end{document}
