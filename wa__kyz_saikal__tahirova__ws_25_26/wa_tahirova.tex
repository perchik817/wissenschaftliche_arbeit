\documentclass[a4paper,12pt]{article}

% --- Sprach- und Kodierungseinstellungen ---
\usepackage[utf8]{inputenc}
\usepackage[T1]{fontenc}
\usepackage[ngerman]{babel}

% --- Typografie ---
\usepackage{lmodern}
\usepackage{microtype}
\usepackage{csquotes} % Korrekte deutsche Anführungszeichen
\usepackage{setspace}
\onehalfspacing

% --- Seitenlayout ---
\usepackage[left=2.5cm,right=2.5cm,top=3cm,bottom=3cm]{geometry}
\usepackage{indentfirst}
\setlength{\parindent}{1.25em} % Отступ первой строки
\setlength{\parskip}{0pt}      % Без дополнительного вертикального расстояния

% --- Grafiken ---
\usepackage{graphicx}
\graphicspath{{./img/}} 

% --- Farben, Links, URLs ---
\usepackage{xcolor}
\usepackage{hyperref}
\usepackage{url}
\urlstyle{same}
\hypersetup{
    pdftitle={Nutzung von Sprachmodellen zur Verbesserung von Bezeichnern in Java-Programmen},
    pdfauthor={Kyz Saikal Tahirova},
    pdfsubject={Wissenschaftliche Arbeit},
    pdfkeywords={Sprachmodelle, Java, Bezeichner, Programmierung},
    colorlinks=true,
    linkcolor=black,
    citecolor=black,
    urlcolor=blue,
    pdfpagemode=UseOutlines
}

\usepackage{enumitem}
\setlist{itemsep=3pt, topsep=3pt, parsep=0pt, partopsep=0pt}

% --- Mathe und Tabellen ---
\usepackage{amsmath,amssymb,booktabs}

% --- Quellcode ---
\usepackage{listings}
\lstdefinestyle{javaStyle}{
    language=Java,
    basicstyle=\ttfamily\small,
    keywordstyle=\color{blue!70!black}\bfseries,
    stringstyle=\color{red!60!black},
    commentstyle=\color{gray},
    numbers=left,
    numberstyle=\tiny\color{gray},
    stepnumber=1,
    showstringspaces=false,
    tabsize=4,
    breaklines=true,
    frame=single,
    inputencoding=utf8,
    extendedchars=true,
    literate={ä}{{\"a}}1 {ö}{{\"o}}1 {ü}{{\"u}}1 {ß}{{\ss}}1
}
\lstset{style=javaStyle}

% --- Verzeichnisse ---
\usepackage[nottoc,notlof,notlot]{tocbibind}

% ===============================
%           Dokument
% ===============================
\begin{document}

% --- Titelseite ---
\begin{titlepage}
    \centering
    \includegraphics[width=0.7\textwidth]{WHZ-Logo.jpg}\\[1cm]

    {\Large\textbf{Westsächsische Hochschule Zwickau}}\\[1ex]
    {\large Fakultät für Physikalische Technik und Informatik}\\[4cm]

    {\LARGE\textbf{Wissenschaftliche Arbeit}}\\[1cm]

    {\Large\textbf{Thema:}}\\[0.5cm]
    {\Large Nutzung von Sprachmodellen zur Verbesserung von Bezeichnern in Java-Programmen}\\[3cm]

    \begin{tabular}{rl}
        Vorgelegt von: & Kyz Saikal Tahirova \\
        Matrikel-Nr.:  & 66110                 \\
        Studiengang:   & Informatik            \\
        Betreuer:      & Prof.\ Dr.\ Laue      \\
        Abgabetermin:  & xx.xx.2025            \\
    \end{tabular}
\end{titlepage}

\setcounter{page}{2}

\clearpage

% --- Inhaltsverzeichnis ---
\tableofcontents
\clearpage

% --- Hauptteil ---
\section*{Kapitel 1}
\section{Einleitung}

In dieser Arbeit wird untersucht, wie große Sprachmodelle (LLMs) automatisch die Namen von Variablen, Methoden und Klassen in Java verbessern können. Verständliche Bezeichner erleichtern das Lesen und Warten von Code und sind entscheidend für die Softwarequalität \cite{oracle}. In der Praxis treten jedoch häufig kurze, unklare oder konventionswidrige Namen auf, die die Lesbarkeit und Wartbarkeit einschränken.

Ein zentrales Problem sind sogenannte \textit{linguistic anti-patterns} \cite{arnaoudova2014}: wiederkehrende sprachliche Muster in Bezeichnern, die zu Missverständnissen führen oder Java-Konventionen verletzen. Dazu gehören etwa generische Namen wie \texttt{data} oder \texttt{temp}, Abkürzungen, die schwer zu interpretieren sind, oder inkonsistente Schreibweisen. Die Arbeit untersucht, ob LLMs in der Lage sind, solche Fehler automatisch zu erkennen und bessere, verständliche Namen vorzuschlagen.

Ziel ist es, eine weitgehend automatische Lösung zu entwickeln:
\begin{enumerate}
    \item Erstellung von Java-Beispielen mit schlechten Bezeichnern.
    \item Übermittlung dieser Beispiele an ein Sprachmodell.
    \item Automatische oder manuelle Auswertung der Vorschläge nach Verständlichkeit, Bedeutungsnähe und Einhaltung von Java-Konventionen.
\end{enumerate}

Dabei wird insbesondere geprüft, welche Art von \textit{Prompts} \cite{startup2023} am effektivsten ist, um konventionsgerechte und verständliche Bezeichner zu erzeugen und sprachliche Anti-Patterns zu vermeiden.

\begin{quote}
\textbf{Forschungsfrage:} Wie gut können Sprachmodelle automatisch bessere Bezeichner in Java vorschlagen, ohne dass ein Mensch eingreifen muss?
\end{quote}

Die Arbeit ist in mehrere Kapitel gegliedert:

\begin{enumerate}
    \item \textbf{Kapitel 1 (Einleitung).} Vorstellung des Themas, Zielsetzung, Relevanz von Sprachmodellen in der Softwareentwicklung sowie die Forschungsfrage.
    \item \textbf{Kapitel 2 (Theoretische Hintergrund).} Einführung in Java-Bezeichner, Namenskonventionen und linguistische Anti-Patterns.
    \item \textbf{Kapitel 3 (Analyse vorhandener Arbeiten).} Eine Überprüfung der Forschung zur Codeverbesserung basierend auf KI zeigt Ansätze, Ergebnisse und bestehende Lücken, insbesondere in Bezug auf Anti-Patterns. 
    \item \textbf{Kapitel 4 (Methodik und Durchführung).} Beschreibung der Vorgehensweise zur Untersuchung von Sprachmodellen, Erstellung von Java-Beispielen mit schlechten Bezeichnern und Auswertung der Ergebnisse. Eine automatische Pipeline wird nur theoretisch erwähnt.  
    \item \textbf{Kapitel 5 (Evaluation).}. Die Ergebnisse zeigen, wie gut Sprachmodelle Vorschläge hinsichtlich Verständlichkeit, Bedeutung, Konventionen und Vermeidung von Anti-Patterns liefern.  
    \item \textbf{Kapitel 6 (Diskussion und Schluss).} Grenzen der Methode, Verbesserungsideen, Ausblick auf weitere Forschung.
    
\end{enumerate}

Hinweis: In dieser ersten Version der Arbeit wird der automatische Pipeline-Ansatz nur theoretisch erläutert, da die Umsetzung zum jetzigen Zeitpunkt noch nicht erfolgt ist. In einer späteren Version könnte diese Funktion ergänzt werden, um den gesamten Prozess der Bezeichnerverbesserung automatisch abzubilden.


\section*{Kapitel 2}
\section{Theoretische Hintergrund}
\subsection{Java-Bezeichner}

Java-Bezeichner sind die Namen von Variablen, Methoden, Klassen, Schnittstellen, Konstanten und Paketen. Sie sollen die Funktion und den Zweck des Codes verständlich machen. Gute Bezeichner sind selbsterklärend, konsistent und folgen klaren Regeln \cite{oracle}. Dies erleichtert das Lesen, Verstehen und Warten von Software.

\subsection{Namenskonventionen in Java}

Java verwendet für Bezeichner verschiedene Schreibweisen, die hauptsächlich auf CamelCase basieren. Dabei gilt:

\begin{itemize}
    \item {Klassen und Schnittstellen.} Namen sind Substantive, jede Wortanfangsbuchstabe wird großgeschrieben, z.B. \texttt{Student}, \texttt{Scanner}, \texttt{Runnable}. Abkürzungen und Akronyme sollten vermieden werden.
    \item {Methoden.} Namen sind Verben, beginnen mit einem Kleinbuchstaben, innere Wortanfänge werden großgeschrieben, z.B. \texttt{calculateSum()}, \texttt{main()}. 
    \item {Variablen.} Namen sollten kurz, aber aussagekräftig sein. Keine Unterstriche oder Dollarzeichen am Anfang. Ein-Zeichen-Variablen nur temporär (z.B. \texttt{i, j, k}). Beispiel: \texttt{totalScore}, \texttt{marks}.
    \item {Konstanten.} Alle Buchstaben groß, Wörter durch Unterstriche getrennt, z.B. \texttt{MAX\_SIZE}, \texttt{PI\_VALUE}.
    \item {Pakete.} Alles klein, oft in Anlehnung an Top-Level-Domains wie \texttt{com}, \texttt{org}. Beispiel: \texttt{java.util.Scanner}.
\end{itemize}

Die konsequente Anwendung dieser Regeln erhöht die Verständlichkeit und Wartbarkeit von Code, insbesondere in größeren Projekten.

\subsection{Linguistische Anti-Patterns}

\textit{Linguistic Anti-Patterns (LAs)} beschreiben wiederkehrende schlechte Praktiken in der Benennung, Dokumentation und Implementierung von Softwareelementen. Sie stellen also sprachliche Widersprüche oder Inkonsistenzen zwischen Namen, Kommentaren und tatsächlichem Verhalten einer Methode oder eines Attributs dar \cite{arnaoudova2014}.

Im Gegensatz zu klassischen Design-Anti-Patterns, die strukturelle Probleme betreffen, beziehen sich LAs auf die sprachliche Ebene des Codes. Sie zeigen sich zum Beispiel, wenn eine Methode \texttt{get()} heißt, aber keinen Wert zurückgibt, oder eine Methode \texttt{isValid()} keinen booleschen Wert liefert. Ebenso gehören widersprüchliche Kommentare oder unklare Attributnamen zu häufigen Fällen solcher sprachlichen Inkonsistenzen.

Arnaoudova et al. (2013) \cite{newfamlas} analysierten mehrere Open-Source-Projekte wie ArgoUML, Cocoon und Eclipse, um diese Muster systematisch zu erfassen. Dabei wurden typische Kategorien von LAs identifiziert:
\begin{itemize}
    \item Inkonsistenz zwischen Name und Rückgabewert.
    \item Widersprüche zwischen Kommentar und Implementierung
    \item Mehrdeutige oder irreführende Bezeichner.
    \item Verwendung generischer Namen ohne semantischen Bezug.
\end{itemize}

Das Bewusstsein für linguistische Anti-Patterns hilft, Missverständnisse zu vermeiden und die Lesbarkeit, Verständlichkeit und Wartbarkeit von Software nachhaltig zu verbessern.

\subsection{Große Sprachmodelle (LLMs)}

Große Sprachmodelle (LLM), wie GPT oder CodeBERT, haben den Bereich der automatischen Programmierung grundlegend verändert. Sie ermöglichen nicht nur die Code-Erstellung in natürlicher Sprache, sondern führen auch zunehmend Aufgaben wie die Code-Erklärung, die Testgenerierung oder die Fehlerbehebung durch. Daher verlagert sich der Schwerpunkt der Softwareentwicklung von der manuellen Implementierung auf die Zusammenarbeit zwischen Mensch und Modell.

Wie in \textit{Automatic Programming: Large Language Models
and Beyond} \cite{aitomaticprogramming} beschrieben, besteht jedoch weiterhin ein zentrales Problem in der Vertrauenswürdigkeit automatisch generierten Codes. Obwohl LLMs beeindruckende Ergebnisse liefern, bleibt die Frage offen, wann und unter welchen Bedingungen dieser Code als zuverlässig genug gilt, um in reale Projekte integriert zu werden. Neben der Korrektheit spielen auch Aspekte wie Sicherheit, Nachvollziehbarkeit und rechtliche Verantwortung eine Rolle.

In dieser Arbeit wird untersucht, inwiefern Sprachmodelle diese Aufgaben zuverlässig ausführen und in welchen Fällen menschliches Eingreifen weiterhin notwendig bleibt.


% --- Literaturverzeichnis ---
\clearpage
\begin{thebibliography}{99}
	
    \bibitem{arnaoudova2014}
    Arnaoudova, V.; Di Penta, M.; Antoniol, G. (2014): 
    \textit{Linguistic Anti-Patterns: What They Are and How Developers Perceive Them.} 
    Empirical Software Engineering.  
    Verfügbar unter: \href{https://www.veneraarnaoudova.ca/wp-content/uploads/2014/10/2014-EMSE-Arnaodova-et-al-Perception-LAs.pdf}{veneraarnaoudova.ca}.

    \bibitem{newfamlas}
    Conference: European Conference on Software Maintenance and Reengineering (2013): 
    \textit{A New Family of Software Anti-Patterns: Linguistic Anti-Patterns.}
    Verfügbar unter: \href{https://assets.ptidej.net/Publications/Documents/CSMR13d.doc.pdf}{assets.ptidej.net}
	
    \bibitem{startup2023}
    Startup Creator (2023): 
    \textit{Die besten ChatGPT Prompts.}  
    Verfügbar unter: \href{https://startup-creator.com/blog/die-besten-chatgpt-prompts/}{startup-creator.com}.
	
    \bibitem{oracle}
    Oracle: 
    \textit{Naming Conventions.}  
    Verfügbar unter: \href{https://www.oracle.com/java/technologies/javase/codeconventions-namingconventions.html}{Java Naming Conventions}.

    \bibitem{aitomaticprogramming}
MICHAEL R., LYU; BAISHAKHI, R.; ABHIK, R.; SHIN HWEI TAN; PATANAMON, T.:
    \textit{Automatic Programming: Large Language Models
and Beyond.}
Verfügbar unter: \href{https://dl.acm.org/doi/pdf/10.1145/3708519}{dl.acm.org}
	
\end{thebibliography}

\end{document}
